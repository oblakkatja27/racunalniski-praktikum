\documentclass[11pt]{article} 
\usepackage[a4paper, margin=2.5cm]{geometry}
\usepackage[slovene]{babel}
\usepackage[utf8]{inputenc}
\usepackage[T1]{fontenc}

\usepackage{mathptmx}

\usepackage{amsmath, amssymb, amsfonts, amsthm}
\usepackage{graphicx}

% definicija okolij – OBE sta pokončni (kot v PDF-ju)
\theoremstyle{definition}
\newtheorem{definicija}{Definicija}
\newtheorem{izrek}{Izrek}

\title{Brownovo gibanje}
\author{Matej Rojec}
\date{}

\begin{document}

\maketitle

Brownovo gibanje (več v \cite{karatzas1991brownian}) je intuitivno slučajen proces,
ki predstavlja naključno gibanje delcev v mediju.

\begin{figure}[h!]
    \centering
    \includegraphics[scale=0.3]{PerrinPlot2.pdf}
    \caption{Reprodukcija slike iz Jean Baptiste Perrin, \emph{Mouvement brownien et réalité moléculaire},
    Ann. de Chimie et de Physique (VIII) 18, 5--114, 1909}
\end{figure}

\begin{definicija}
Standardno Brownovo gibanje $\{B_t\}_{t \ge 0}$ je slučajen proces z naslednjimi lastnostmi:
\begin{enumerate}
\item $B_0 = 0$.
\item Prirastki $B_{t_n}-B_{t_{n-1}}, B_{t_{n-1}}-B_{t_{n-2}}, \ldots, B_2-B_1, B_1-B_0$
so neodvisne slučajne spremenljivke, za vsak
$t_0 \le t_1 \le \cdots \le t_{n-1} \le t_n$.
\item Za vsak $t \ge 0$ in $h > 0$ velja $B_{t+h}-B_t \sim \mathcal{N}(0,h)$.
\item Funkcija $t \mapsto B_t$ je zvezna skoraj gotovo.
\end{enumerate}
\end{definicija}

Preden zapišemo izrek, definirajmo še pojem časa ustavljanja.

\begin{definicija}
Slučajna spremenljivka $\tau$ na verjetnostnem prostoru $(\Omega,F,P)$ z vrednostmi v $\mathbb{R}^+$
je čas ustavljanja glede na filtracijo $(F_t)_{t \in T}$, če velja:
$\forall t \in T : \{\tau \le t\} \in F_t$.
\end{definicija}

Zdaj lahko zapišemo izrek 1.

\begin{izrek}
Naj bo $\{B_t\}_{t \ge 0}$ (standardno) Brownovo gibanje, $\tau$ čas ustavljanja glede na
$(F_t)_{t \ge 0}$ in naj velja $P[\tau < \infty] = 1$.
Potem je tudi proces:
\[
\hat B := \{B_{T+t}-B_T \mid t \ge 0\}
\]
(standardno) Brownovo gibanje in neodvisen od $F_T$.
\end{izrek}

\begin{thebibliography}{1}

\bibitem{karatzas1991brownian}
Ioannis Karatzas and Steven E.~Shreve.
\newblock \emph{Brownian Motion and Stochastic Calculus}.
\newblock Springer, 1991.

\end{thebibliography}

\end{document}
